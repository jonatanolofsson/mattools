\chapter{Introduktion}
\label{cha:introduktion}
I den här laborationsrapporten presenteras resultat och förarbete till
systemidentifikationslaborationen i kursen TSRT62 Modellbygge och
simulering, vid Linköpings Universitet. Laborationens syfte är att skapa
en modell och identifiera parametrar för en vindflöjel.

Experimentuppställningen består i en elektrisk motor med en propeller som
blåser på en skiva. Skivan är monterad på gångjärn så att den kan svänga
fritt i vinden längs den horisontella axeln. Den matematiska modellen har
som mål att modellera systemet med motorspänning som insignal till en
spänning, som är beroende av vinkeln på vindflöjeln, som utsignal.

I Kapitel~\ref{cha:experiment} beskrivs identifieringsexperimentens utförande.